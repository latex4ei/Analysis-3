% % % % % % % % % % % % % % % % % % % % % % % % % % % % % % % % % % % % % % % % 
% Formelsammlung von LaTeX4EI									
%
% @encode: 	UTF-8, tabwidth = 4, newline = LF
% @author:	Lukas Kompatscher, Emanuel Regnath (orginal HM3 und HM4)
% @date:		
%
% % % % % % % % % % % % % % % % % % % % % % % % % % % % % % % % % % % % % % % % 

%---------------------------------------%
%				Analysis 3				%
%~~~~~~~~~~~~~~~~~~~~~~~~~~~~~~~~~~~~~~~%

% Document Class ===============================================================
\documentclass[german,color]{latex4ei/latex4ei_fs}

% set document information
\title{Analysis 3}
\author{Lukas Kompatscher (LaTeX4EI)}	
\myemail{info@latex4ei.de}	


% DOCUMENT_BEGIN ===============================================================
\begin{document}

\maketitle

% SECTION ======================================================================
\section{Nützliches Wissen $e^{\i x} = \cos (x) + \i \cdot \sin(x)$}
% ==============================================================================
\begin{sectionbox}
	\subsection{Sinus, Cosinus \quad $\sin^2(x) \bs + \cos^2(x) = 1$}
	\setlength{\tabcolsep}{4pt}
	\begin{tablebox}{c|c|c|c|c||c|c|c|c} \ctrule
	$x$ & $0$ & $\pi / 6$ & $\pi / 4$ & $\pi / 3$ & $\frac{1}{2}\pi$ & $\pi$ & $1\frac{1}{2}\pi$ & $2 \pi$ \\
	$\scriptstyle{ \varphi }$ & $\scriptstyle{0^\circ}$ & $\scriptstyle{30^\circ}$ & $\scriptstyle{45^\circ}$ & $\scriptstyle{60^\circ}$ & $\scriptstyle{90^\circ}$ & $\scriptstyle{180^\circ}$ & $\scriptstyle{270^\circ}$ & $\scriptstyle{360^\circ}$ \\ \cmrule
	$\sin$ & $0$ & $\frac{1}{2}$ & $\frac{1}{\sqrt{2}}$ & $\frac{\sqrt 3}{2}$ & $1$ & $0$ & $-1$ & $0$ \\
	$\cos$ & $1$ & $\frac{\sqrt 3}{2}$ & $\frac{1}{\sqrt 2}$ & $\frac{1}{2}$ & $0$ & $-1$ & $0$ & $1$ \\     
	$\tan$ & $0$ & $\frac{\sqrt{3}}{3}$ &	$1$	&	$\sqrt{3}$ & $\pm \infty$ & $0$ & $\mp \infty$ & $0$\\ \cbrule
	\end{tablebox}
	\begin{tabular*}{\columnwidth}{@{\extracolsep\fill}ll@{}}
		Additionstheoreme &  Stammfunktionen\\
	 	$\cos (x - \frac{\pi}{2}) = \sin x$ & $\int x \cos(x) \diff x = \cos(x) + x \sin(x)$\\
	 	$\sin (x + \frac{\pi}{2}) = \cos x$ & $\int x \sin(x) \diff x = \sin(x) - x \cos(x)$\\
	 	$\sin 2x = 2 \sin x \cos x $  & $\int \sin^2(x) \diff x = \frac12 \bigl(x - \sin(x)\cos(x) \bigr)$\\ 
	 	$\cos 2x = 2\cos^2 x - 1$  & $\int \cos^2(x) \diff x = \frac12 \bigl(x + \sin(x)\cos(x) \bigr)$\\
	 	$\sin(x) = \tan(x)\cos(x)$ & $\int \cos(x)\sin(x) = -\frac12 \cos^2(x)$ \\
	\end{tabular*}\\[1em]
		\textbf{Sinus/Cosinus Hyperbolicus} $\sinh, \cosh$\\ 
		% \quad \operatorname{arsinh}\ x:= \ln\left(x+\sqrt{x^2+1}\right) \\
		\begin{tabular*}{\columnwidth}{@{\extracolsep\fill}ll@{}}
		$\sinh x = \frac{1}{2}(e^x -e^{-x})= - \i \, \sin(\i x)$ & $\cosh^2 x  \bs - \sinh^2 x = 1$\\
		$\cosh x  = \frac{1}{2}(e^x +e^{-x})= \cos(\i x)$ & $\cosh x + \sinh x = e^{x}$\\
		\end{tabular*}\\
		\textbf{Kardinalsinus} $\mathrm{si}(x) = \frac{\sin(x)}{x}$ \qquad genormt: $\sinc(x) = \frac{\sin(\pi x)}{\pi x}$
\end{sectionbox}

% Dokumentende
% ======================================================================
\end{document}